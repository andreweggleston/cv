%%%%%%%%%%%%%%%%%
% This is an sample CV template created using altacv.cls
% (v1.1.2, 1 February 2017) written by LianTze Lim (liantze@gmail.com). Now compiles with pdfLaTeX, XeLaTeX and LuaLaTeX.
%
%% It may be distributed and/or modified under the
%% conditions of the LaTeX Project Public License, either version 1.3
%% of this license or (at your option) any later version.
%% The latest version of this license is in
%%    http://www.latex-project.org/lppl.txt
%% and version 1.3 or later is part of all distributions of LaTeX
%% version 2003/12/01 or later.
%%%%%%%%%%%%%%%%%

%% If you need to pass whatever options to xcolor
\PassOptionsToPackage{dvipsnames}{xcolor}

% !TEX program = lualatex

%% If you are using \orcid or academicons
%% icons, make sure you have the academicons
%% option here, and compile with XeLaTeX
%% or LuaLaTeX.
% \documentclass[10pt,a4paper,academicons]{altacv}

%% Use the "normalphoto" option if you want a normal photo instead of cropped to a circle
\documentclass[10pt,letter,normalphoto]{altacv}

%% AltaCV uses the fontawesome and academicon fonts and packages.
%% See texdoc.net/pkg/fontawecome and http://texdoc.net/pkg/academicons for full list of symbols.
%%
%% Compile with LuaLaTeX for best results. If you
%% want to use XeLaTeX, you may need to install
%% Academicons.ttf in your operating system's font
%% folder.


% Change the page layout if you need to
\geometry{left=1cm,right=9cm,marginparwidth=6.8cm,marginparsep=1.2cm,top=1.25cm,bottom=1.25cm}

% Change the font if you want to.
  % If using pdflatex:
  \usepackage[utf8]{inputenc}
  \usepackage[T1]{fontenc}
  \usepackage[default]{lato}
  % charter, helvet, newcent

  % If using xelatex or lualatex:
  %  \setmainfont{lato}

% Change the colours if you want to
\definecolor{Purp}{HTML}{000000}%FFE059}
\definecolor{SlateGrey}{HTML}{000000}%FF5964}
\definecolor{LightGrey}{HTML}{2E2E00}
\definecolor{DarkBlue}{HTML}{000000}%143D5D}
\definecolor{LightBlue}{HTML}{000000}%35A7FF}

\colorlet{heading}{Purp}
\colorlet{heading-line}{LightBlue}
\colorlet{accent}{DarkBlue}
\colorlet{emphasis}{SlateGrey}
\colorlet{body}{LightGrey}

\ifdefined\iscolor
\definecolor{HeadingColor}{HTML}{143D5D}%FFE059}
\definecolor{TitleColor}{HTML}{DE9E36}%FF5964}
\definecolor{BodyColor}{HTML}{37323E}%2E2E00}
\definecolor{LinkColor}{HTML}{6D6A75}%143D5D}
\definecolor{LightBlue}{HTML}{DEB841}%35A7FF}

\colorlet{heading}{HeadingColor}
\colorlet{heading-line}{LightBlue}
\colorlet{accent}{LinkColor}
\colorlet{emphasis}{TitleColor}
\colorlet{body}{BodyColor} % !!!! If you want color, uncomment this!
\fi

% Change the bullets for itemize and rating marker
% for \cvskill if you want to
\renewcommand{\itemmarker}{{\small\textbullet}}
\renewcommand{\ratingmarker}{\faCircle}

%% sample.bib contains your publications
% \addbibresource{sample.bib}

\begin{document}
\firstname{Andrew  }
\lastname{Eggleston}
\tagline{Software Engineer}
%\photo{2.8cm}{profile-square}
\personalinfo{%
  % You can add your own with \printinfo{symbol}{detail}
  \email{\href{mailto:egglestonandrew927@gmail.com}{egglestonandrew927@gmail.com}}
  \phone{\href{tel:15089885104}{+1 (508) 988-5104}}
  \github{\href{https://www.github.com/andreweggleston}{andreweggleston}}
}

%% Make the header extend all the way to the right, if you want.
\begin{fullwidth}
  \makecvheader
\end{fullwidth}

%% Provide the file name containing the sidebar contents as an optional parameter to \cvsection.
%% You can always just use \marginpar{...} if you do
%% not need to align the top of the contents to any
%% \cvsection title in the "main" bar.


\cvsection[resume-sidebar]{Work Experience}

  \cveventnew{\href{https://www.northropgrumman.com/Pages/default.aspx}{Northrop Grumman}}{Software Engineer}{Created an employee management database. Gained experience in \textbf{Java EE}, \textbf{angular.js}, and \textbf{SQL}. Utilized Agile software development process to organize tasks.}{June 2019 - August 2019}{Baltimore, MD}

  \divider

  \cveventnew{\href{https://www.genelec.com}{Genelec Inc.}}{Technician}{Designed systems for routing audio and video from production to editing. Created network architecture using \textbf{Dante} protocol and DHCP. Maintained DSP calibration hardware installed in speakers.}{March 2020 - Present}{Natick, MA}


  % \divider



\cvsection{Projects}
  \cvevent{Death By Dagger}{\href{https://github.com/andreweggleston/DeathByDagger}{github.com/andreweggleston/DeathByDagger}}{}{}
  A fun game, updated for the electronic age. Made use of the \textbf{Slack API} to keep players updated on their status in the game.
  Created using \textbf{Go}, with data hosted on \textbf{PostgreSQL}.

  \divider

  \cvevent{Javascript Raytracing}{\href{https://github.com/andreweggleston/js-rtx}{github.com/andreweggleston/js-rtx}}{}{}
  A small \textbf{p5.js} example created to demonstrate how raytracing can be used to make a depth effect when rendering images (as long as everything can be described as a straight line).
  Created during the design process of a two day hackathon.
  
  \divider

  \cvevent{Observer}{\href{https://github.com/andreweggleston/observer}{github.com/andreweggleston/observer}}{}{}
  A web application designed to view Dota 2 replays. Replays consist of \textbf{ProtoBuf} network packets from all players. \textbf{Observer} translates this network data into meaningful information such as player location and state. Written in \textbf{Go}, frontend with \textbf{React}.
  \divider

  \cvevent{WHS Planner}{\href{https://github.com/s0urc3d3v3l0pm3nt/whs\_planner}{github.com/s0urc3d3v3l0pm3nt/whs\_planner}}{}{}
  A client application developed to provide students with a simple schedule planner and organizer.
  The application features an intuitive calendar, an automatically updating schedule, and a live news feed.
  The app was created using \textbf{Java}, with \textbf{JavaFX} as a graphical framework.



\medskip

\end{document}
